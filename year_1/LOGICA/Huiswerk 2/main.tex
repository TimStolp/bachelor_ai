%%%%%%%%%%%%%%%%
%% Preambule  %%
%%%%%%%%%%%%%%%%

% De preambule bestaat uit alles tot aan het '\begin{document}' commando. Doorgaans staan hier commando's die op het hele bestand van toepassing zijn. Voor nu staan er alvast wat pakketten en commando's die handig zijn voor later.

\documentclass[11pt]{article}         

\usepackage{amsmath,amsfonts,amssymb} 
\usepackage{upgreek}                  
\usepackage{enumerate}             
\usepackage{multicol}    
\usepackage{fitch}
\usepackage[all]{xy}                
\usepackage{tikz-qtree}               
\usepackage[margin=2.5cm]{geometry} 

\usepackage{fancyhdr}             
\usepackage{lastpage}               
\setlength{\parindent}{0pt}        

\pagestyle{fancy}                 

\lhead{\opdrachtNaam\ \opdrachtNummer}           
\rhead{\naam(\studentNummer)}               
\rfoot{Pagina\ \thepage\ van\ \pageref{LastPage}}
\lfoot{\datum}                              
\cfoot{}                                        

\renewcommand\headrulewidth{0.4pt}  
\renewcommand\footrulewidth{0.4pt}

\newcommand{\E}{\exists} 
\newcommand{\A}{\forall}

\newcommand{\ccen}[2]{\llap{$#1$}${}\mathrel{\circ}{}$\rlap{$#2$}}

%%%%%%%%%%%%%
%%  Titel  %%
%%%%%%%%%%%%%

% Zie thuisopdracht.

%%%%%%%%%%%%%%
%% Gegevens %%
%%%%%%%%%%%%%%

% Vul hier je gegevens in.

\newcommand{\naam}          {Tim Stolp}
\newcommand{\studentNummer} {11848782}
\newcommand{\opdrachtNaam}  {Huiswerk}
\newcommand{\opdrachtNummer}{3}
\newcommand{\datum}         {20-11-17}

%%%%%%%%%%%%%%%%
%% Antwoorden %%
%%%%%%%%%%%%%%%%

\begin{document}

\section*{Opgave 5.31}


\begin{fitch}
\fh A A A & Assumption \\
\fa B B B & Main Proof Step \\
\fa C C C & Another Main Proof Step\\
\fa \fh D D D & New Assumption \\
\fa \fa E E E & Subproof Step \\
\fa \fa F F F & Another Subproof Step \\
\fa G G G & Main Proof Step \\
\fa H H H & Main Proof Step
\end{fitch}






\end{document} 

